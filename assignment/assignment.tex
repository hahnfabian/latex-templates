\documentclass[12pt]{report}
\usepackage[margin=1in]{geometry} 
\usepackage[english]{babel}                     % Change language to adjust the date format/ index language
\usepackage{amsmath,amsthm,amssymb,scrextend}
\usepackage{fancyhdr}
\pagestyle{fancy}
\usepackage{graphicx} 
\usepackage[myheadings]{fullpage}
\usepackage{fancyhdr}
\usepackage{lastpage}
\usepackage{graphicx, wrapfig, subcaption, setspace, booktabs}
\usepackage[T1]{fontenc}
\usepackage[font=small, labelfont=bf]{caption}
\usepackage{fourier}
\usepackage[protrusion=true, expansion=true]{microtype}
\usepackage[spanish]{babel}
\usepackage{makecell}
\usepackage{float}
\renewcommand{\thesection}{\arabic{section}}
\usepackage{hyperref}
\usepackage{amsmath}
\usepackage{advdate}


\hypersetup{
    hidelinks
}
\newcommand{\HRule}[1]{\rule{\linewidth}{#1}}
\newcommand\pef[1]{[\ref{#1}]}
\onehalfspacing
\setcounter{tocdepth}{5}
\setcounter{secnumdepth}{5}

\begin{document}
\title{ \normalsize \textsc{Course Name Part 1} \\ Course Name Part 2
		\\ [2.0cm]
		\HRule{0.5pt} \\
		\LARGE \textbf{\uppercase{Name Line 1} \\ Name Line 2}
		\HRule{2pt} \\ [0.5cm]
		\normalsize \today \vspace*{5\baselineskip}}

\date{}

\author{
		Author 1 \\
        Author 2
  }

\maketitle

\tableofcontents

\pagestyle{fancy}
\fancyhf{}
\newpage
\lhead{Short Name}
\chead{Authors maybe here}
\rhead{ \today}

\section{Some section 1}

La tasa de crecimiento del componente gigante \pef{fig:tasadecreci} se aproxima al valor 1 con un crecimiento logístico (en este punto todos los nodos estarían contenidos en el componente gigante). A partir del punto crítico en un valor medio de enlaces de 1, la tasa de crecimiento aumenta rápidamente y se aplana antes de alcanzar $y=1$.

\begin{figure}[H]
    \centering
    \includegraphics[width=0.5\linewidth]{picture.png}
    \caption{Label here}
    \label{fig:tasadecreci}
\end{figure}



\end{document}



